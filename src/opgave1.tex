\section*{Opgave 1}

\subsection*{Algoritme 1}
Jeg ser først på følgende algoritme:
\begin{algorithm}
	\caption{Løkke1($n$)}
	\label{løkke1}
	\begin{algorithmic}[1]
		\For{$i=1$ to $n$}
			\For{$j=1$ to $n+1-i$}
				\State {\bf print} $i+j$
			\EndFor
		\EndFor
	\end{algorithmic}
\end{algorithm}

Da linje 3 er den eneste linje, som ikke er en kontrolstruktur, er det den eneste jeg ser på.
Jeg kan opskrive matematisk hvor mange gange linje 3 kører:
\begin{equation}
	\label{math1}
	\sum_{i=1}^{n}\sum_{j=1}^{n+1-i}1
\end{equation}

Den ydre summering repræsenterer den ydre løkke og den indre summering den indre løkke.
Vi ved, at følgende gælder:
\begin{equation*}
	\sum_{i=1}^{n}1 = n
\end{equation*}

Det kan jeg benytte til, at finde værdien af den indre summering:
\begin{equation*}
	\sum_{j=1}^{n+1-i}1 = n+1-i
\end{equation*}

Resultatet af det kan jeg da indsætte i \eqref{math1}:
\begin{equation}
	\label{math2}
	\sum_{i=1}^{n} \left(n+1-i\right)
\end{equation}

Vi ved yderligere, at følgende gælder:
\begin{equation*} \begin{split}
	\sum_{i=1}^{n}i & = \frac{n(n+1)}{2}\\
	\sum_{i=1}^{n}n & = n^2
\end{split} \end{equation*}

Det kan jeg benytte til, at finde værdien af \eqref{math2}:
\begin{equation} \begin{split}
	\label{math3}
	\sum_{i=1}^{n} \left(n+1-i\right) & = n^2 + n - \frac{n(n+1)}{2}\\
	                                  & = n^2 + n - \frac{1}{2}n^2 - \frac{1}{2}n\\
	                                  & = \frac{1}{2}n^2 + \frac{1}{2}n
\end{split} \end{equation}

Det er kun det første led, der er relevant her. Ud fra dette kan jeg opskrive:
\begin{equation*}
	\text{Løkke1}\left(n\right) = O\left(n^2\right)
\end{equation*}

Det gælder, da der findes en konstant $c$ således, at $c\ n^2$ før eller siden vil blive større end $\frac{1}{2}n^2 + \frac{1}{2}n$.
Bestemmer konstanten $c$:
\begin{equation*} \begin{split}
	c\cdot n^2 > \frac{1}{2}n^2 + \frac{1}{2} &\quad\Rightarrow\quad c > \frac{\frac{1}{2}n^2}{n^2} + \frac{\frac{1}{2}n}{n^2}\\
	                                          &\quad\Rightarrow\quad c > \frac{1}{2} + \frac{1}{2n}
\end{split} \end{equation*}

Jeg ser på et tilfælde, hvor $c\ n$ bliver større end $\frac{1}{2}n^2 + \frac{1}{2}n$ ved $n=1$:
\begin{equation*}
	c > \frac{1}{2} + \frac{1}{2\cdot 1} \quad\Rightarrow\quad c > 1
\end{equation*}

Efterprøver med $c=2$ ved $n=1$
\begin{equation*}
	2\cdot 1^2 > \frac{1}{2} + \frac{1}{2\cdot 1}
\end{equation*}

Jeg kan også opskrive følgende ud fra \eqref{math3}:
\begin{equation*}
	\text{Løkke1}\left(n\right) = \Omega\left(n^2\right)
\end{equation*}

Det gælder, da der findes en konstant $c$ således, at $c\ n^2$ før eller siden vil blive mindre end $\frac{1}{2}n^2 + \frac{1}{2}n$.
Bestemmer konstanten $c$:
\begin{equation*} \begin{split}
	c\cdot n^2 < \frac{1}{2}n^2 + \frac{1}{2} &\quad\Rightarrow\quad c < \frac{\frac{1}{2}n^2}{n^2} + \frac{\frac{1}{2}n}{n^2}\\
	                                          &\quad\Rightarrow\quad c < \frac{1}{2} + \frac{1}{2n}
\end{split} \end{equation*}

\subsection*{Algoritme 2}
Jeg ser nu på følgende algoritme:
\begin{algorithm}
	\caption{Løkke2($n$)}
	\label{løkke2}
	\begin{algorithmic}[1]
		\State $i = 1$
		\While{$i < n$}
			\For{$j=1$ to $n$}
				\State $i = i + 1$
			\EndFor
			\State $i = i\cdot 2$
		\EndWhile
	\end{algorithmic}
\end{algorithm}