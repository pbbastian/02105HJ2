\section*{Opgave 1}

\subsection*{Algoritme 1}
Jeg ser først på følgende algoritme:
\begin{algorithm}
	\caption{Løkke1($n$)}
	\label{løkke1}
	\begin{algorithmic}[1]
		\For{$i=1$ to $n$}
			\For{$j=1$ to $n+1-i$}
				\State {\bf print} $i+j$
			\EndFor
		\EndFor
	\end{algorithmic}
\end{algorithm}

Da linje 3 er den eneste linje, som ikke er en kontrolstruktur, er det den eneste jeg ser på.
Jeg kan opskrive matematisk hvor mange gange linje 3 kører:
\begin{equation}
	\label{math1}
	\sum_{i=1}^{n}\sum_{j=1}^{n+1-i}1
\end{equation}

Den ydre summering repræsenterer den ydre løkke og den indre summering den indre løkke.
Vi ved, at følgende gælder:
\begin{equation*}
	\sum_{i=1}^{n}1 = n
\end{equation*}

Det kan jeg benytte til, at finde værdien af den indre summering:
\begin{equation*}
	\sum_{j=1}^{n+1-i}1 = n+1-i
\end{equation*}

Resultatet af det kan jeg da indsætte i \eqref{math1}:
\begin{equation}
	\label{math2}
	\sum_{i=1}^{n} \left(n+1-i\right)
\end{equation}

Vi ved yderligere, at følgende gælder:
\begin{equation*} \begin{split}
	\sum_{i=1}^{n}i & = \frac{n(n+1)}{2}\\
	\sum_{i=1}^{n}n & = n^2
\end{split} \end{equation*}

Det kan jeg benytte til, at finde værdien af \eqref{math2}:
\begin{equation} \begin{split}
	\label{math3}
	\sum_{i=1}^{n} \left(n+1-i\right) & = n^2 + n - \frac{n(n+1)}{2}\\
	                                  & = n^2 + n - \frac{1}{2}n^2 - \frac{1}{2}n\\
	                                  & = \frac{1}{2}n^2 + \frac{1}{2}n
\end{split} \end{equation}

Det er kun det første led, der er relevant her. Ud fra dette kan jeg opskrive:
\begin{equation*}
	\text{Løkke1}\left(n\right) = \Theta\left(n^2\right)
\end{equation*}

\subsection*{Algoritme 2}
Jeg ser nu på følgende algoritme:
\begin{algorithm}
	\caption{Løkke2($n$)}
	\label{løkke2}
	\begin{algorithmic}[1]
		\State $i = 1$
		\While{$i < n$}
			\For{$j=1$ to $n$}
				\State $i = i + 1$
			\EndFor
			\State $i = i\cdot 2$
		\EndWhile
	\end{algorithmic}
\end{algorithm}

Man kunne i princippet lige så godt skrive linje 4 som $i = i + n$, da $i$ bliver forøget med 1 $n$ gange.\\

Betingelsen for while-løkken på linje 2 bliver allerede brudt ved første iteration,
da $i = i + n$ medfører, at $i = 1 + n$ (da $i = 1$ gælder ved løkkens start),
hvilket selvfølgelig medfører, at $i > n$. Derved gælder $i < n$ ikke længere.\\

Altså køres linje 1 og 6 kun én gang. Linje 4 køres $n$ gange. Det giver os:
\begin{equation*}
	\text{Løkke2}\left(n\right) = n + 3
\end{equation*}

Ud fra det kan jeg opskrive følgende:
\begin{equation*}
	\text{Løkke2}\left(n\right) = O(n)
\end{equation*}

Dette gælder, da der findes en konstant $c$ således, at $c\cdot n > n$ altid gælder. Denne konstant $c$
kunne f.eks. være $2$.
\begin{equation*}
	\text{Løkke2}\left(n\right) = \Omega(n)
\end{equation*}

Dette gælder, da der findes en konstant $c$ således, at $c\cdot n < n$ altid gælder. Denne konstant $c$
kunne f.eks. være $\frac{1}{2}$. Jeg kan nu konkludere følgende:
\begin{equation*}
	\text{Løkke2}\left(n\right) = \Theta(n)
\end{equation*}